The human brain and the human gut microbiome
are complex, dynamic, and \hl{something} systems
that are functionally linked through the gut-brain-microbiome axis.
\hl{add stuff about communication between systems}.

Both the brain and the microbiome also have crucial
early developmental windows in the first years of life.
The brain reaches \hl{XX\%} of its adult size
and near-adult levels of myelenation by age 5,
and also establishes gross patterns of neuronal connections
by the same age.
The microbiome undergoes rapid waves of succession after birth,
transitioning from a sterile or near-sterile environment during gestation,
to low-diversity, Actinobacteria and Proteobacteria-dominated communities
while consuming primarily formula or breast milk,
to high-diversity, more adult like Bacteroidetes/Firmicutes-dominated
communities after the transition to solid food.
Emerging evidence suggests that the compositiona and function
of the gut microbiome can have profound effects on neural development,
yet despite the intimate communication
and critical early periods of development shared by these two systems,
relatively little is know about the interplay
between the gut microbiome and the brain in early life.
