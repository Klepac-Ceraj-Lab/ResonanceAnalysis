\subsection*{Outline}

\begin{itemize}
  \item Intial description: Cohort size, demographics, data types
  \item Basic analysis: Summary statistics, ordinations, permanovas, mantel tests
  \item Exploration of taxonomic profiles + cogscores / brain, cross-sectional. Linear models. \hl{Is this necessary??}
  \item Try HAllA on multivariate data?
  \item Predictive model (RF) - taxonomic profiles + cogscores.
  \item Taxonomic profiles + brain structure
  \item Gene functions + metabolomes
  \item Topo data analysis?
\end{itemize}

\subsection*{Cohort and sample collection}

In order to investigate the normal codevelopment of the brain and the microbiome,
we recruited \hl{XX} mothers with uncomplicated pregnancies
and followed children from \hl{YY} to \hl{ZZ} months of age
(Figure 1 - see Materials and Methods).
\hl{I wonder if we should put an upper limit on ages (maybe 5 years)
to reduce complications in this section.
It only loses us a couple of dozen.}
DNA from stool samples was extracted and shotgun sequenced
\hl{(\mu XX reads per sample)}
yielding taxonomic and gene-function profiles.
A subset of stool samples were also analyzed by LCMS,
yielding metabolic profiles.
Coincident with stool sample collection,
extensive surveys were performed to assess family and medical history,
and children were evaluated using age-appropriate assessments
of cogntive development.
A subset of children were also scanned using magnetic resonance imaging (MRI).

\subsection*{Variation in the microbiome is associated with neurocogntitive development}

When all samples were analyzed together,
overall variation in microbial taxa and gene functions
explained \hl{XX\%} and \hl{YY\%} of variation in cognitive function scores
assessed at the same time as stool samples were collected
\hl{adjusted p-value < XX, < YY respectively}. 
However, there is a marked change in microbiome composition
after the introduction of solid foods,
such that the age at which stool samples were collected
explained a large fraction (\hl{ZZ\% and AA\%} of overall variation.
We therefore stratified our study population into those
less than 6 months of age, all of whom remain on liquid diets,
and those over 12 months of age, all of whom have begun consuming solid foods.




\subsection*{Variation in the microbiome predicts neurocogntitive development}